\documentclass[aspectratio=169,12pt,t]{beamer}
\usepackage{graphicx}
\setbeameroption{hide notes}
\setbeamertemplate{note page}[plain]
\usepackage{listings}
\usepackage{ulem}

% header.tex: boring LaTeX/Beamer details + macros

% get rid of junk
\usetheme{default}
\beamertemplatenavigationsymbolsempty
\hypersetup{pdfpagemode=UseNone} % don't show bookmarks on initial view


% font
\usepackage{fontspec}
\setsansfont
  [ ExternalLocation = fonts/ ,
    UprightFont = *-regular ,
    BoldFont = *-bold ,
    ItalicFont = *-italic ,
    BoldItalicFont = *-bolditalic ]{texgyreheros}
\setbeamerfont{note page}{family*=pplx,size=\footnotesize} % Palatino for notes
% "TeX Gyre Heros can be used as a replacement for Helvetica"
% I've placed them in fonts/; alternatively you can install them
% permanently on your system as follows:
%     Download http://www.gust.org.pl/projects/e-foundry/tex-gyre/heros/qhv2.004otf.zip
%     In Unix, unzip it into ~/.fonts
%     In Mac, unzip it, double-click the .otf files, and install using "FontBook"

% named colors
\definecolor{offwhite}{RGB}{255,250,240}
\definecolor{gray}{RGB}{155,155,155}
\definecolor{purple}{RGB}{177,13,201}
\definecolor{green}{RGB}{46,204,64}

\definecolor{background}{RGB}{255,255,255}
\definecolor{foreground}{RGB}{24,24,24}
\definecolor{title}{RGB}{27,94,134}
\definecolor{subtitle}{RGB}{22,175,124}
\definecolor{hilit}{RGB}{122,0,128}
\definecolor{vhilit}{RGB}{255,0,128}
\definecolor{codehilit}{RGB}{255,0,128}
\definecolor{lolit}{RGB}{95,95,95}
\definecolor{myyellow}{rgb}{1,1,0.7}
\definecolor{nhilit}{RGB}{128,0,128}  % hilit color in notes
\definecolor{nvhilit}{RGB}{255,0,128} % vhilit for notes

\newcommand{\hilit}{\color{hilit}}
\newcommand{\vhilit}{\color{vhilit}}
\newcommand{\nhilit}{\color{nhilit}}
\newcommand{\nvhilit}{\color{nvhilit}}
\newcommand{\lolit}{\color{lolit}}

% use those colors
\setbeamercolor{titlelike}{fg=title}
\setbeamercolor{subtitle}{fg=subtitle}
\setbeamercolor{institute}{fg=lolit}
\setbeamercolor{normal text}{fg=foreground,bg=background}
\setbeamercolor{item}{fg=foreground} % color of bullets
\setbeamercolor{subitem}{fg=lolit}
\setbeamercolor{itemize/enumerate subbody}{fg=lolit}
\setbeamertemplate{itemize subitem}{{\textendash}}
\setbeamerfont{itemize/enumerate subbody}{size=\footnotesize}
\setbeamerfont{itemize/enumerate subitem}{size=\footnotesize}

% page number
\setbeamertemplate{footline}{%
    \raisebox{5pt}{\makebox[\paperwidth]{\hfill\makebox[20pt]{\lolit
          \scriptsize\insertframenumber}}}\hspace*{5pt}}

% add a bit of space at the top of the notes page
\addtobeamertemplate{note page}{\setlength{\parskip}{12pt}}

% default link color
\hypersetup{colorlinks, urlcolor={hilit}}

\lstset{language=bash,
        basicstyle=\ttfamily\scriptsize,
        frame=single,
        commentstyle=,
        backgroundcolor=\color{offwhite},
        showspaces=false,
        showstringspaces=false
        }


% a few macros
\newcommand{\bi}{\begin{itemize}}
\newcommand{\bbi}{\vspace{24pt} \begin{itemize} \itemsep8pt}
\newcommand{\ei}{\end{itemize}}
\newcommand{\be}{\begin{enumerate}}
\newcommand{\bbe}{\vspace{24pt} \begin{enumerate} \itemsep8pt}
\newcommand{\ee}{\end{enumerate}}
\newcommand{\sbi}{\begin{itemize} \fontsize{9pt}{9.5}\selectfont}
\newcommand{\sbe}{\begin{enumerate} \fontsize{9pt}{9.5}\selectfont}
\newcommand{\ig}{\includegraphics}
\newcommand{\subt}[1]{{\footnotesize \color{subtitle} {#1}}}
\newcommand{\ttsm}{\tt \small}
\newcommand{\ttfn}{\tt \footnotesize}
\newcommand{\figh}[2]{\centerline{\includegraphics[height=#2\textheight]{#1}}}
\newcommand{\figw}[2]{\centerline{\includegraphics[width=#2\textwidth]{#1}}}


%%%%%%%%%%%%%%%%%%%%%%%%%%%%%%%%%%%%%%%%%%%%%%%%%%%%%%%%%%%%%%%%%%%%%%
% end of header
%%%%%%%%%%%%%%%%%%%%%%%%%%%%%%%%%%%%%%%%%%%%%%%%%%%%%%%%%%%%%%%%%%%%%%

% title info
\title{BMI 881}
\subtitle{Biomedical Data Science Scholarly Literature 1}
\author{\href{https://kbroman.org/BMI881}{\tt kbroman.org/BMI881} }
\institute{}
\date{\small \hspace{3in} Karl Broman \\
  \hspace{3in} \href{https://kbroman.org}{\color{foreground}
    \small \tt kbroman.org}}



\begin{document}

\begin{frame}[c]{Which do you prefer? {\hilit (vote by writing on the slide)}}

\renewcommand{\arraystretch}{2}
  \begin{tabular}{|c|c|c|c|c|} \cline{1-2} \cline{4-5}
    Zoom            &\hspace{1in} && Google Meet        & \hspace{1in}  \\ \cline{1-2} \cline{4-5}
    BBCollaborate   &&& Microsoft Teams                 & \\ \cline{1-2} \cline{4-5}
    WebEx           &&& They are all fine  & \\ \cline{1-2} \cline{4-5}
    Skype           &&& They are all bad   & \\ \cline{1-2} \cline{4-5}
                    &&&                    & \\ \cline{1-2} \cline{4-5}
                    &&&                    & \\ \cline{1-2} \cline{4-5}
   \end{tabular}

\end{frame}


% title slide
{
\setbeamertemplate{footline}{} % no page number here
\frame{
  \titlepage
} }




\begin{frame}{Goals}

  \bbi
\item Read and discuss a bunch of journal articles
\item Critical evaluation of articles
\item Practice talking about data science
\item Get to know each other
  \ei

\end{frame}




\begin{frame}{The technology}

  \bbi
\item Use your video camera if possible
\item Keep yourself muted
\item ``Raise your hand'' to talk
\item Unmute when called on
\item Use chat for brief comments, and sparingly
  \ei

\end{frame}


\begin{frame}{Responsibilities}

  \bbi
\item Read the papers in advance
\item Participate in the discussions
\item Let me know if you need to miss a class
\item Tuesdays:
  \bi
  \item 2-paragraph summary of papers to be discussed that day
  \item Plus 2 or more questions for discussion
  \ei
\item A couple of other written assignments during the semester
  \ei

\end{frame}



\begin{frame}{Tuesday summaries}

  \bbi
\item One paragraph summarizing the work
\item One paragraph with your reaction
\item Two or more questions for discussion
\item Posted someplace that I can get them:
  \bi
\item Blog
\item Github repository
\item Box or Dropbox or Google Drive folder
\item Email them to me (least preferred)
  \ei
\item Post by 8:00am each Tuesday
\ei

\end{frame}


\begin{frame}{Office hours}

  \bbi
\item Tuesday 9:30-10:30am
\item Friday 9:30-10:30am
\item Via BBCollaborate, \href{https://bit.ly/offhrs_f2020}{\tt bit.ly/offhrs\_f2020}
\item \hilit or by appointment
\ei

\end{frame}






\begin{frame}{}

  \begin{columns}
    \column{0.5\textwidth}

      \centerline{\color{title} Norms for discussion}

      \bigskip

      {\footnotesize
      \bi
    \item Presume positive intentions
    \item Engage respectfully
    \item Listen attentively
    \item Aim for equal participation
    \item Respect boundaries
    \item Provide evidence
      \ei }


    \column{0.5\textwidth}

      \centerline{\color{title} Online working agreements}

      \bigskip

      \footnotesize

      \bi
    \item Use your video camera if possible
    \item Use names to address each other
    \item Use features (mute, raise hand, chat, etc) appropriately
    \item Be okay with silence
    \item Start and end on time
      \ei

  \end{columns}

\bigskip \bigskip \bigskip
\bigskip \bigskip \bigskip

\hfill {\scriptsize from \href{https://discussion.education.wisc.edu}{The Discussion Project}}

\end{frame}



\begin{frame}{Tips on handling of technical problems?}

\end{frame}




\begin{frame}[c]{}

\centerline{\Large \color{title} How do you read a journal article?}

\end{frame}



\begin{frame}[c]{How to acquire a journal article}

  \bbi
\item Search \href{https://scholar.google.com}{Google Scholar} and click ``All \uline{\hspace{2em}} versions''
\item Install the
  \href{https://unpaywall.org/products/extension}{unpaywall browser
    extension}
\item Paste {\tt ezproxy.library.wisc.edu} into the URL for the
  article
\item Use the
  \href{https://kb.wisc.edu/helpdesk/page.php?id=68164}{GlobalProtect
    VPN for UW-Madison}
\item \href{https://sci-hub.se}{Sci-Hub} provides pirated versions of many
  journal articles \\
  ({\vhilit not} acceptable)
\item I'll post PDFs on the \href{https://canvas.wisc.edu}{Canvas site}
  \ei

\end{frame}



\end{document}
